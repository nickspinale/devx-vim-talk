\documentclass{beamer}
\mode<presentation>{\usetheme{Frankfurt}}

\usepackage[english]{babel}
\usepackage[utf8]{inputenc}
\usepackage{times}
\usepackage[T1]{fontenc}

\usepackage{amsmath}
\usepackage[all]{xy}

\title{Group Theory's Applications to \\ Solving the Rubik's Cube Blindfolded}
\author{Nicholas C.~Spinale}
\institute{Department of Mathematics \\ Flathead Valley Community College}
\date{FVCC Student Research Conference, 2014}

\pgfdeclareimage[height=0.5cm]{university-logo}{fvcc_seal_color.jpg}
\logo{\pgfuseimage{university-logo}}

\AtBeginSubsection
	{
	\begin{frame}<beamer>{Outline}
	\tableofcontents[currentsubsection]
	\end{frame}
	}

\begin{document}
		
				\begin{frame}
				\titlepage
				\end{frame}
		
		\section*{Introduction}

				\begin{frame}{The Problem}
					Example: \href{https://www.youtube.com/watch?v=mCyYPimImyM}{\beamergotobutton{Link}}
					\begin{itemize}
						\item Rules: look then solve
						\item Inherent division: memorization and execution
						\item Aim for a balance
					\end{itemize}
					Misconceptions
					\begin{table}
						\centering \footnotesize
						\begin{tabular}{|c||c|} \hline
							\small Memorization & \small Execution \\[.1cm] \hline \hline
							Nothing & Peak... \\ \hline
							Memorize the cube's position & Solve parallel to the "brain cube" \\ \hline
							Figure out and memorize a solution & Execute memorized solution \\ \hline
							Nothing & Repeat magical algorithm until solved \\ \hline
						\end{tabular}
					\end{table}
					\textbf{My message:} blindfolded solving does \emph{NOT} necessarily involve cheating, \emph{NOR} a photographic memory
				\end{frame}

				\begin{frame}{My Plan}
				\textbf{My message:} blindfolded solving does \emph{NOT} necessarily involve cheating, \emph{NOR} a photographic memory \\[.4cm]
				\tableofcontents
				\end{frame}

		\section{Essential Group Theory}

			\subsection{Background}
		
				\begin{frame}{The Big Picture}
					\begin{itemize}
						\item Mathematical Objects
						\begin{itemize}
							\item Sets
							\item Functions
							\item Relations
							\item Quantities
						\end{itemize}
						\item Mathematics Structure: a set and one more more relevant objects
						\item Axioms define structures of particular interest to allow for generalizations
						\item Group Theory is the study of one such structure			
					\end{itemize}
				\end{frame}

			\subsection{What is a Group?}

				\begin{frame}{What is a Group?}
					 For set \(G\) and binary operator \, \(*:G \, x \, G \to G\) \\
					\((G,*)\) is a group if, for any \(a,b,c \in G\)	
					\begin{table}
						\centering
						\begin{tabular}{r|p{7cm}}
							Closure & \(a*b \in G\) \\[.3cm]
							Associativety & \((a*b)*c = a*(b*c)\) \\[.3cm]
							Identity & There exists some \(e \in G\) such that, for all \(g \in G, \, e*g = g*e = g\) \\[.3cm]
							Inverse & For all \(g \in G\), there exists some \(g^{-1}\) such that \(g*g^{-1} = g^{-1}*g = e\) \\
						\end{tabular}
					\end{table}
				\end{frame}
				
				\begin{frame}{An Example}
					\((\mathbb{R},*)\) \: The real numbers under multiplication
					\begin{table}
						\centering
						\begin{tabular}{r|p{7cm}}
							Closure & \(-6.7432*\frac{19}{3425} \in \mathbb{R}\) \\[.3cm]
							Associativety & \((6*2)*7 = 6*(2*7)\) \\[.3cm]
							Identity & 1*43 = 43*1= 43 \\[.3cm]
							Inverse & \(3*\frac{1}{3} = 1\) \\
						\end{tabular}
					\end{table}
					\emph{NOTE:} Not all groups are commutative!
				\end{frame}
		
		\section{A New Perspective on the Cube}

				\begin{frame}{WE ARE READY!}
					\begin{center}
					\includegraphics[width = 7.2cm, height = 7.5cm]{thecube.png}
					\end{center}
				\end{frame}

			\subsection{Necessary Info}

				\begin{frame}{Some Important Notes}
					We will never move the centers! \\[.3cm]
					Groups of stickers belong to pieces, so \\[.1cm]
					\( \quad \quad \) Corner stickers and edge stickers remain as such \\[.1cm]
					\( \quad \quad \) Each sticker has \emph{EXACTLY 1} correct position \\[.3cm]
					States vs. effects \\[.4cm]
					There are \( 18 = 6*3 \) legal primative effects
					\[R,L,U,D,F,B\]
					\[R^{-1},L^{-1},U^{-1},D^{-1},F^{-1},B^{-1}\]
					\[R^{2},L^{2},U^{2},D^{2},F^{2},B^{2}\]
				\end{frame}
			
			\subsection{The Cube Group}

				\begin{frame}{The Cube Group}
					Let set \(X_{C}\) contain the 24 corner sticker locations\\
					Let set \(X_{E}\) contain the 24 edge sticker locations\\[.3cm]
					Consider a pair of bijective maps ("re-arrangements")
					\[ \sigma : X_C \to X_C \text{  and  } \tau : X_E \to X_E \]  \\[.2cm]
					\( \sigma \) assigns a new corner location to each corner location \\
					\( \tau \) assigns a new edge location to each edge location \\[.2cm]
					All elements of \( G = \, <R,L,U,D,F,B> \), \\
					the set of legal effects on a state of the cube, \\
					are composed of one such \( \sigma \) and one such \( \tau \)
					\[ G = \{ \, (\sigma , \tau) \,| \,\sigma \in A(X_C), \tau \in A(X_C) \,\} \]
				\end{frame}

				\begin{frame}{Example: \( F = (\sigma_F ,\tau_F) \)}
					\begin{columns}[c]
						\column{.7\textwidth}
						\includegraphics[width = 8cm]{FPERM.png} \\
						\( \sigma_F \scriptstyle = (\,I\,J\,K\,L\,)\,(\,D\,M\,V\,G\,)\,(\,C\,P\,U\,F\,) = (\,I\,J\,K\,L\,) = \dots \) \\
						\( \tau_F \scriptstyle = (\,I\,J\,K\,L\,)\,(\,C\,P\,U\,F\,) = (\,I\,J\,K\,L\,) = (\,C\,P\,U\,F\,) \)
						\column{.3\textwidth}
						\includegraphics[height = 7cm]{fpermtable.png}
					\end{columns}
				\end{frame}
	
				\begin{frame}{The Cube Group: \( (g,*) \)}
					\(G\) is a group under "\(*\)" (composition)
					\[ a = (\sigma_1,\tau_1), b = (\sigma_2,\tau_2) \implies a*b = (\sigma_1 \circ \sigma_2, \tau_1 \circ \tau_2) \]
					\[ (\upsilon_{1} \circ \upsilon_{2}) (a) = \upsilon_{2} (\upsilon_{1} (a))\] \\
					\begin{table}
						\centering
						\begin{tabular}{r|p{7cm}}
							Closure & \( a,b \in G \implies a*b \in G \) \\[.3cm]
							Associativety & \( (a*b)*c = a*(b*c) \) \\[.3cm]
							Identity & Let \((e,e)\) map every location to itself \\[.3cm]
							Inverse & \(a = (\sigma, \tau) \implies a^{-1} = (\sigma^{-1}, \tau^{-1}) \) \\
						\end{tabular}
					\end{table}
				\end{frame}

		\section{Blindfolded Cubing}

			\subsection{The General Strategy}
		
				\begin{frame}{The General Strategy}
					Phase 1: Memorization \\[.1cm]
					\quad \quad After observing cube scrambled by \( s = (\sigma_s,\tau_s) \)  \\
					\quad \quad find effect \( m = (\sigma_m,\tau_m) \) \\ 
					\quad \quad such that \( s * m = (\sigma_s \circ \sigma_m, \tau_s \circ \tau_m) = (e,e) \) \\[.5cm]
				\pause
					Phase 2: Execution \\
					\begin{itemize}
						\item \emph{SIMPLIFY}: make small alterations\\
						\item Group theory tells us that smallest are 3-cycles of either corners or edges\\
						\item For \( \sigma_m \) and \( \tau_m \) seperately, \\
							use 3-cycles to reduce each permutation to \\ 
							either \(e\) or a double transposition
						\item If necessary, deal with parity \\[.3cm]
					\end{itemize}
				\end{frame}

			\subsection{Execution Techniques}
		
				\begin{frame}{Execution Techniques}
					Creating 3-cycles \\[.2cm]
					\begin{itemize}
						\item Commutators: \quad \( aba^{-1}b^{-1} \) \quad denoted \( [\,a\,;b\,]\)
						\item Conjugates: \quad \( aba^{-1} \) \quad denoted \( <a\,;b> \) \\[.7cm]
					\end{itemize}
					Dealing with Parity
				\end{frame}

				\begin{frame}{Easy-ization}
					Ways to make the above approach practical for a human \\[.2cm]
					\begin{itemize}
						\item Buffer spots
						\item Stefan Pochmann
						\item M2 and R2 methods
						\item The classical approach
					\end{itemize}
				\end{frame}
		
		\section*{Conclusion}

				\begin{frame}{Conclusion}
					Parting Words \\[.2cm]
					\begin{itemize}
						\item I can give you resources to learn more about group theory
						\item I can give you resources to learn more about the cube
						\item If enough people want to come, I can demonstrate a blindfolded solve
					\end{itemize}
				\end{frame}

				\begin{frame}{Thank You}
					\begin{center}
					\includegraphics[width = 7.2cm, height = 7.5cm]{thecube.png}
					\end{center}
				\end{frame}

	\appendix

		\section{\appendixname}
			
			\subsection{Print}
				
				\begin{frame}[allowframebreaks]
				\frametitle<presentation>{Print}
				\begin{itemize}
				\item Halmos, Paul R. (1960). Naïve Set Theory: The University Series in Undergraduate Mathematics. New York, NY: D. Van Nostrand Company.
				\item Herstein, I. N. (1999). Abstract Algebra. Third Ed. New York, NY: John Wiley and Sons, Inc.
				\item Livio, Mario. (2006). The Equation that Couldn’t Be Solved: How Mathematical Genius Discovered the Language of Symmetry. New York, NY: Simon and Schuster, Inc.
				\item Niven, Ivan and Zuckerman, Herbert S. (1966). An Introduction to the Theory of Numbers. Second Ed. New York, NY: John Wiley and Sons, Inc.
				\item Struik, Dirk J. (1967). A Concise History of Mathematics. Third Ed. New York, NY: Dover Publications, Inc.
				\item Tabak, John. (2004). Algebra: Sets, Symbols  and the Language of Thought. New York, NY: Facts on File, Inc.
				\item Wilder, Raymond L. (1968). Evolution of Mathematical Concepts: An Elementary Study. New York, NY: John Wiley and Sons, Inc.
				\end{itemize}
				\end{frame}

			\subsection{Non-encyclopedia Internet Sources}

				\begin{frame}[allowframebreaks]{Non-encyclopedia Internet Sources}
				\begin{itemize}
				\item Burris, Stanley and Sankappanavar, H. P. Stanley. (1981). A Course in Universal Algebra. Retrieved May 25, 2013 from
				\item Chen, Janet. Group Theory and the Rubik’s Cube. Retrieved May 8, 2013 from Harvard University Department of Mathematics website: http://www.math.harvard.edu/~jjchen/docs/Group%20Theory%20and%20the%20Rubik's%20Cube.pdf
				\item Clark, Edwin W. (December 23, 2001). Elementary Abstract Algebra. Retrieved May 20, 2013 from University of South Florida website: http://www.math.usf.edu/~eclark/Elem\_abs\_alg.pdf
				\item Comte, Auguste. (May 15, 2012). The Philosophy of Mathematics. Retrieved June 10, 2013 from Project Gutenburg website: http://www.gutenberg.org/3/9/7/0/39702/
				\item D’Mello, Cherie and Weiss, William (1997). Fundamentals of Model Theory. Retrieved June 5, 2013 from University of Toronto Department of Mathematics website: http://www.math.uwo.ca/~mdawes/courses/420/mod\_th.pdf
				\item Garron, Lucas. (June 8, 2011). An Introduction to Group Theory and the Rubik's Cube. Retrieved May 15, 2013 from Stanford Mathematics Professor Lucas Garron’s website: http://stanford.garron.us/class/math120/files/math120\_wim.pdf
				\item Grizzard, Philip R. (January 31, 2007). Introduction to Group Theory. Retrieved May 15, 2013 from University of Illinois at Chicago website: www.math.uic.edu/~grizzard/Professional/WCgroup.pdf
				\item Isaksen, Carl Joakim (June 2012). Rubik’s Cube and Group Theory. Retrieved June 10, 2013 from the University of Oslo Department of Mathematics and Natural Sciences: https://www.duo.uio.no/bitstream/handle/10852/10881/MasterthesisCarlJoakimIsaksen.pdf
				\item Markwig, Thomas. (February 2009). Algebraic Structures. Retrieved May 20, 2013 from the University of Kaiserslautern department of mathematics website: http://www.mathematik.uni-kl.de/~keilen/download/LectureNotes/algebraicstructures.pdf
				\item MIT Department of Mathematics. (March 17, 2009). Mathematics of the Rubik’s Cube: Introduction to Group Theory and Permutation Puzzles. Retrieved May 14, 2013 from the MIT Department of Mathematics website: http://web.mit.edu/sp.268/www/rubik.pdf
				\item Sato, Naoki. (May 2000). Number Theory. Retrieved May 9, 2013 from http://www.artofproblemsolving.com/Resources/Papers/SatoNT.pdf.
				\item Scherphuis, Jaap. (n.d.) Computational Group Theory. Retrieved May 25, 2013 from : http://www.jaapsch.net/puzzles/schreier.htm
				\item Scherphuis, Jaap. (n.d.) Group Theory for Puzzles. Retrieved May 25, 2013 from : http://www.jaapsch.net/puzzles/groups.htm
				\item Scherphuis, Jaap. (n.d.) Useful Mathematics. Retrieved May 25, 2013 from : http://www.jaapsch.net/puzzles/theory.htm
				\item Warner, Stefan. (July 2003). Introduction to Group Theory. Retrieved May 15, 2013 from Hofstra University website: http://people.hofstra.edu/faculty/Stefan\_Waner/RealWorld/pdfs/145Notes.pdf
				\end{itemize}
				\end{frame}

			\subsection{Online Encyclopedia Sources}

				\begin{frame}[allowframebreaks]
				\frametitle<presentation>{Online Encyclopedia Sources}
				\begin{itemize}
				\item Algebraic structure. (n.d.). In Wikipedia, the Free Encyclopedia. Retrieved May 25, 2013, from Wikipedia.com: http://en.wikipedia.org/
				\item Axiom. (n.d.). In Wikipedia, the Free Encyclopedia. Retrieved June 10, 2013, from Wikipedia.com: http://en.wikipedia.org/
				\item Foundations of mathematics. (n.d.). In Wikipedia, the Free Encyclopedia. Retrieved June 7, 2013, from Wikipedia.com: http://en.wikipedia.org/
				\item Galois theory. (n.d.). In Wikipedia, the Free Encyclopedia. Retrieved May 22, 2013, from Wikipedia.com: http://en.wikipedia.org/
				\item Group theory. (n.d.). In Wikipedia, the Free Encyclopedia. Retrieved May 8, 2013, from Wikipedia.com: http://en.wikipedia.org/
				\item Intuitionistic type theory. (n.d.). In Wikipedia, the Free Encyclopedia. Retrieved June 5, 2013, from Wikipedia.com: http://en.wikipedia.org/
				\item Invariant (mathematics). (n.d.). In Wikipedia, the Free Encyclopedia. Retrieved June 5, 2013, from Wikipedia.com: http://en.wikipedia.org/
				\item Mathematical Atlas. (n.d.). Information gathered from the following branch-categories: Number theory, Algebra areas of mathematics, and Group theory and generalizations, Lattices. Retrieved June 7, 2013 from http://www.math-atlas.org/
				\item Mathematical object. (n.d.). In Wikipedia, the Free Encyclopedia. Retrieved June 7, 2013, from Wikipedia.com: http://en.wikipedia.org/
				\item Model Theory. (July 20, 2009). In the Stanford Encyclopedia of Philosophy. Retrieved June 12, 2013 from plato.stanford.edu: http://plato.stanford.edu/entries/model-theory/
				\item Outline of algebraic structures. (n.d.). In Wikipedia, the Free Encyclopedia. Retrieved May 25, 2013, from Wikipedia.com: http://en.wikipedia.org/
				\item Symmetry in mathematics. (n.d.). In Wikipedia, the Free Encyclopedia. Retrieved June 5, 2013 from Wikipedia.com: http://en.wikipedia.org/
				\end{itemize}
				\end{frame}

\end{document}



%				\begin{frame}{Permutations}
%					Let \(X\) be a finite set \\[.8cm]
%					Consider \( \sigma \), a bijection that maps \(X\) onto itself:
%					\[ \sigma : X \to X \text{ | for any } a,b \in X, a \neq b \implies \sigma(a) \neq \sigma(b) \] \\[.5cm]
%					\( \sigma \) is called a permutaion of \(X\)
%				\end{frame}
%				
%				\begin{frame}[allowframebreaks]{Example: the permutations of \( \mathbb{Z}_{3} \) }
%					\(\mathbb{Z}_{3}\) has 3! permutations: \\ \[A( \mathbb{Z}_{3} ) = \{ \sigma_{1},\sigma_{2},\sigma_{3},\sigma_{4},\sigma_{5},\sigma_{6} \} \] 
%					\[\sigma_{1}(1) = 1\quad \sigma_{1}(2) = 2\quad \sigma_{1}(3) = 3 \]
%					\[\sigma_{2}(1) = 1\quad \sigma_{2}(2) = 3\quad \sigma_{2}(3) = 2 \]
%					\[\sigma_{3}(1) = 2\quad \sigma_{3}(2) = 1\quad \sigma_{3}(3) = 3 \]
%					\[\sigma_{4}(1) = 2\quad \sigma_{4}(2) = 3\quad \sigma_{4}(3) = 1 \]
%					\[\sigma_{5}(1) = 3\quad \sigma_{5}(2) = 2\quad \sigma_{5}(3) = 1 \]
%					\[\sigma_{6}(1) = 3\quad \sigma_{6}(2) = 1\quad \sigma_{6}(3) = 2 \]
%					\framebreak
%					\[A( \mathbb{Z}_{3} ) = \{ \sigma_{1},\sigma_{2},\sigma_{3},\sigma_{4},\sigma_{5},\sigma_{6} \} \] 
%					\[\sigma_{1}: \quad 1 \mapsto 1\quad  2 \mapsto 2\quad 3 \mapsto 3\]
%					\[\sigma_{2}: \quad 1 \mapsto 1\quad  2 \mapsto 3\quad 3 \mapsto 2\]
%					\[\sigma_{3}: \quad 1 \mapsto 2\quad  2 \mapsto 1\quad 3 \mapsto 3\]
%					\[\sigma_{4}: \quad 1 \mapsto 2\quad  2 \mapsto 3\quad 3 \mapsto 1\]
%					\[\sigma_{5}: \quad 1 \mapsto 3\quad  2 \mapsto 1\quad 3 \mapsto 2\]
%					\[\sigma_{6}: \quad 1 \mapsto 3\quad  2 \mapsto 2\quad 3 \mapsto 1\]
%					\framebreak
%					\[A( \mathbb{Z}_{3} ) = \{ \sigma_{1},\sigma_{2},\sigma_{3},\sigma_{4},\sigma_{5},\sigma_{6} \}, \quad x \in X \] 
%					\begin{table}
%						\centering
%						\begin{tabular}{r||r|r|r|r|r}
%							\( x \) & \( \sigma_{1}(x) \) &  \( \sigma_{2}(x) \) &  \( \sigma_{3}(x) \) & \( \sigma_{4}(x) \) & \( \cdots \) \\ \hline
%							1 & 1 & 1 & 2 & 2 & \( \cdots \) \\
%							2 & 2 & 3 & 1 & 3 & \( \cdots \) \\
%							3 & 3 & 2 & 3 & 1 & \( \cdots \) 
%						\end{tabular}
%					\end{table}
%				\end{frame}
%
%				\begin{frame}{A Speedy, Intuitive Explanation}
%					Let \(X\) be a set of locations \\[.5cm]
%					Interpret \( \sigma \in A(X)\) as a rule that \\ moves the contents of \( a \in X \) to \( \sigma (a) \) \\[.5cm]
%					\( \sigma \) "induces" a permutation on \\ the contents of the locations in \( X \) \\[.5cm]
%					Cycles and notation
%				\end{frame}
%				
%				\begin{frame}{The Symmetric Group}
%					\(A(X)\) is a group under a binary operation called "composition" \\[.3cm]
%					\( \text{For any } a \in X \text{ and } \sigma_{1},\sigma_{2} \in A(X),\)
%					\[ (\sigma_{1} \circ \sigma_{2}) (a) \text{ is defined as } \sigma_{2} (\sigma_{1} (a))\] \\
%					Intuitively, composition is sequential rearranging
%					\begin{table}
%						\centering
%						\begin{tabular}{r|p{7cm}}
%							Closure & \( \sigma_{1} , \sigma_{2} \in A(X) \implies (\sigma_{1} \circ \sigma_{2} ) \in A(X) \) \\[.3cm]
%							Associativety & ( \( \sigma_{1} \circ \sigma_{2}) \circ \sigma_{3} = \sigma_{1} \circ ( \sigma_{2} \circ \sigma_{3}) \) \\[.3cm]
%							Identity & If \( \; \sigma: x_{1} \mapsto x_{2} \; \) then \( \;\sigma ^{-1} : x_{2} \mapsto x_{1} \) \\[.3cm]
%							Inverse & \(e:  x \mapsto x \; \text{ for all } \; x \in X\) \\
%						\end{tabular}
%					\end{table}
%				\end{frame}
				